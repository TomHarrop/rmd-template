\institute[]{
    \href{https://melbournebioinformatics.org.au}{melbournebioinformatics.org.au}
    }

% Enable lato (installed on unix with `tlmgr install lato`)
\PassOptionsToPackage{no-math}{fontspec}
\usepackage[no-math]{fontspec}
\defaultfontfeatures{Ligatures=TeX}
\usepackage[defaultsans]{lato}
\usepackage[utf8]{inputenc}

% Nicer code font
\setmonofont{inconsolata}

% enable dashes and quotation marks
% \defaultfontfeatures{Ligatures=TeX}


% title fonts
\setbeamerfont{title}{parent=lato}
\setbeamerfont{subtitle}{size=\small,parent=lato}
\setbeamerfont{author}{size=\scriptsize,parent=lato}
\setbeamerfont{institute}{size=\scriptsize,parent=lato}
\setbeamerfont{date}{size=\scriptsize,parent=lato}

% Paste emojis as graphics
% Have to use latexincludegraphics because includegraphics has been put in a float
\DeclareRobustCommand{\emojione}{%
  \begingroup\normalfont
    \latexincludegraphics[height=\fontcharht\font`\B]{img/bullet.png}%
  \endgroup
}

\DeclareRobustCommand{\inlinepng}[1]{%
  \begingroup\normalfont
  \latexincludegraphics[height=\fontcharht\font`\B]{#1}%
  \endgroup
}


% Enable citations in footnotes
\usepackage{calc}
\usepackage[absolute,overlay]{textpos}
\setlength{\TPHorizModule}{1in + \hoffset + \oddsidemargin + \textwidth}
\setlength{\TPVertModule}{\headheight + \textheight}
\newcommand{\source}[1]{
  \tiny
  \begin{textblock}{}[1, 0](1, 1)
    \hfill
    \usebeamercolor[fg]{normal text} #1
  \end{textblock}}

\newcommand{\sourceleft}[1]{
  \tiny
  \begin{textblock}{}[0, 1](0, 1)
    \hspace{1in + \oddsidemargin}
    \usebeamercolor[fg]{normal text} #1
  \end{textblock}}

% Remove beamer template
\setbeamertemplate{frametitle}
{\begin{centering}\smallskip
    \insertframetitle\par
    \smallskip\end{centering}}
% \setbeamertemplate{itemize item}{$\bullet$}
\setbeamertemplate{itemize item}{\emojione}
\setbeamertemplate{itemize subitem}{$\bullet$}
\setbeamertemplate{itemize subsubitem}{$\circ$}
\setbeamertemplate{navigation symbols}{}
\setbeamercolor{normal text}{fg=black}

% Centre all figures
\usepackage{letltxmacro}
% Save the meaning of \includegraphics
\usepackage{graphicx}
\LetLtxMacro\latexincludegraphics\includegraphics
% Update the include graphics command to include centering
\renewcommand{\includegraphics}[2][]%
    {\begin{center}
    \latexincludegraphics[#1]{#2}
    \end{center}}

% Define beamer colours
\definecolor{textcolour}{HTML}{797979}
\definecolor{headingcolour}{HTML}{797979}
\definecolor{linkcolour}{HTML}{440154}
\colorlet{paleheadingcolour}{headingcolour!50!}

% Place colours
\setbeamercolor{title}{fg=headingcolour}
\setbeamercolor{frametitle}{fg=headingcolour}
% \setbeamercolor{normal text}{fg=textcolour}
\setbeamercolor{itemize item}{fg=linkcolour} % bullets
\setbeamercolor{itemize subitem}{fg=linkcolour} % sub-bullets
\setbeamercolor{itemize subsubitem}{fg=linkcolour} % sub-sub-bullets
\setbeamercolor{block title}{fg=headingcolour,bg=linkcolour}
\setbeamercolor{block body}{fg=headingcolour,bg=linkcolour}
\setbeamercolor{alerted text}{fg=linkcolour}

% colour both section and url links
% see https://tex.stackexchange.com/a/401885
\hypersetup{
  colorlinks,
  allcolors=.,
  urlcolor=linkcolour,
}

% enable columns
\def\begincols{\begin{columns}}
\def\begincol{\begin{column}}
\def\endcol{\end{column}}
\def\endcols{\end{columns}}

% show layout
\usepackage{layout}


% Set up title page
% from https://tex.stackexchange.com/a/3927
% \usepackage{graphicx}   # already loaded above
\usepackage{tikz}

\setbeamertemplate{title page}{
\begin{tikzpicture}[remember picture,overlay]
  \node[at=(current page.center)] {
    \latexincludegraphics[keepaspectratio,
                          width=\paperwidth,
                          height=\paperheight]{style/title_slide_template.eps}
};
\end{tikzpicture}
\begin{center}
\usebeamerfont{title}\inserttitle\par
\usebeamerfont{subtitle}\insertsubtitle\par
\usebeamerfont{date}\insertdate\par
\usebeamerfont{author}\insertauthor\par
\usebeamerfont{institute}\insertinstitute\par
\end{center}
}


% TOC slides
% \setbeamercolor{section title}{fg=linkcolour,bg=white}
% \setbeamercolor{section}{fg=linkcolour,bg=white}
% \setbeamercolor{section name}{fg=linkcolour,bg=white}
\setbeamercolor{part title}{fg=linkcolour,bg=white}
\setbeamercolor{section in toc}{fg=linkcolour,bg=white}
\setbeamercolor{section in toc shaded}{fg=linkcolour,bg=white}
\setbeamertemplate{section in toc}{
  \inserttocsectionnumber.~\underbar{\inserttocsection}
}
\setbeamertemplate{section in toc shaded}{
  \inserttocsectionnumber.~\inserttocsection
}
\setbeamertemplate{section page}{
  \begin{tikzpicture}[remember picture,overlay]
    \node[at=(current page.center)] {
      \latexincludegraphics[
        keepaspectratio,
        width=\paperwidth,
        height=\paperheight]{style/title_slide_template.eps}
  };
  \end{tikzpicture}
  \begin{center}
  \usebeamerfont{title}\inserttitle\par
  \begin{beamercolorbox}[sep=12pt,center]{part title}
    \usebeamerfont{author}\tableofcontents[currentsection]
  \end{beamercolorbox}
  \end{center}
}

% slide number footer
\setbeamertemplate{footline}[frame number]

% SI units
\usepackage[binary-units]{siunitx}
\AtBeginDocument{%
    \DeclareSIUnit\basepair{bp}%
}
\AtBeginDocument{%
    \DeclareSIUnit\base{b}%
}

% maths fonts with mathastext
% this works, but digits are rendered in the normal math font (not Lato)
% \usepackage[italic,defaultmathsizes,symbolgreek]{mathastext}
% \renewcommand{\familydefault}{\sfdefault}

% maths fonts with unicode-math
% allows you to change font, but *only* to a supported font
% https://www.ctan.org/pkg/unicode-math?lang=en
% \usepackage{unicode-math}
% \setmathfont{STIXTwoMath-Regular}

% Below is the "proper" way to do it. It works, in that digits and letters are
% rendered in Lato, but symbols get clobbered
% UPDATE: this works, but you need 
% --variable=mathspec in the pandoc arguments.
\usepackage{mathspec}
\setmathfont(Digits,Latin,Greek)[Scale=MatchLowercase]{Lato}


% for graphs and networks
% see https://tex.stackexchange.com/a/85700
\makeatletter % undo the wrong changes made by mathspec
\let\RequirePackage\original@RequirePackage
\let\usepackage\RequirePackage
\makeatother
\usepackage{tikz-network}
\tikzset{
  network invisible/.code={%
    \def\VertexFillOpacity{0}
    \def\VertexLineOpacity{0}
    \def\VertexTextOpacity{0}
    \def\EdgeOpacity{0}
    \def\EdgeTextFillOpacity{0}
    \def\EdgeTextOpacity{0}},
  network visible on/.code={%
    \alt#1{}{\tikzset{network invisible}}}}

% colours for highlights
\definecolor{viridis1}{HTML}{440154}
\definecolor{viridis2}{HTML}{472D7B}
\definecolor{viridis3}{HTML}{3B528B}
\definecolor{viridis4}{HTML}{2C728E}
\definecolor{viridis5}{HTML}{21908C}
\definecolor{viridis6}{HTML}{27AD81}
\definecolor{viridis7}{HTML}{5DC863}
\definecolor{viridis8}{HTML}{AADC32}
\definecolor{viridis9}{HTML}{FDE725}

% hilight on slides in ovspec (#1), see beamer manual for \temporal
% use like this
% \begin{scope}[network hili2 on={<2->}{viridis3}]
\tikzset{
  network hili/.code={%
      \def\VertexFillColor{#1}
      \def\VertexLineColor{#1}
      \def\EdgeColor{#1}
      \def\TextColor{#1}%
  },
  network hili on/.code args={#1#2}{%
    \temporal#1{}{%
        \tikzset{network hili=#2}
    }{}}}


% title on slides
% \setbeamertemplate{headline}{
%   \ifnum \theframenumber=1
%   
%   \else
%   \vspace{1ex}
%   \hspace{1in + \hoffset + \oddsidemargin}
%   \usebeamercolor[fg]{normal text}
%   \insertdate :
%   \inserttitle
%   \hfill
%   \fi}


