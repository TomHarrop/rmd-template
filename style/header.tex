
%%%%%%%%%%%%
% METADATA %
%%%%%%%%%%%%

\institute[]{
    \href{https://usegalaxy.org.au}{usegalaxy.org.au}
    \and
    \href{https://melbournebioinformatics.org.au}{melbournebioinformatics.org.au}
    }

\title{Genome annotation with Funannotate}

\newif\ifplacelogo % create a new conditional
\placelogotrue % set it to true

\logo{
\begin{tikzpicture}[remember picture, overlay]
\node[right=0.5cm] at (current page.154){
    \latexincludegraphics[width=1cm]{img/Galaxy-Aust-Logo-Stacked-RGB.pdf}
};
\end{tikzpicture}
}
% \latexincludegraphics[width=1cm]{img/Galaxy-Aust-Logo-Stacked-RGB.pdf}


%%%%%%%%%
% FONTS %
%%%%%%%%%

% Enable lato (installed on unix with `tlmgr install lato`)
\PassOptionsToPackage{no-math}{fontspec}
\usepackage[no-math]{fontspec}
\defaultfontfeatures{Ligatures=TeX}
\usepackage[defaultsans]{lato}
\usepackage[utf8]{inputenc}

% Nicer code font
\setmonofont{inconsolata}

% enable dashes and quotation marks
% \defaultfontfeatures{Ligatures=TeX}

% adjustbox allows clipping and trimming
% \usepackage[export]{adjustbox}

% title fonts
\setbeamerfont{title}{parent=lato}
\setbeamerfont{subtitle}{size=\small,parent=lato}
\setbeamerfont{author}{size=\scriptsize,parent=lato}
\setbeamerfont{institute}{size=\scriptsize,parent=lato}
\setbeamerfont{date}{size=\scriptsize,parent=lato}

%%%%%%%%%%%%%%%%%
% ENABLE EMOJIS %
%%%%%%%%%%%%%%%%%

% Paste emojis as graphics
% Have to use latexincludegraphics because includegraphics has been put in a float
\DeclareRobustCommand{\emojione}{%
  \begingroup\normalfont
    \latexincludegraphics[height=\fontcharht\font`\B]{img/bullet.png}%
  \endgroup
}

\DeclareRobustCommand{\inlinepng}[1]{%
  \begingroup\normalfont
  \latexincludegraphics[height=\fontcharht\font`\B]{#1}%
  \endgroup
}

%%%%%%%%%%%%%%%%%%%%%%%%%%
% DEFINE SOURCE COMMANDS %
%%%%%%%%%%%%%%%%%%%%%%%%%%

% Enable citations in footnotes
\usepackage{calc}
\usepackage[absolute,overlay]{textpos}
\setlength{\TPHorizModule}{1in + \hoffset + \oddsidemargin + \textwidth}
\setlength{\TPVertModule}{\headheight + \textheight}
\newcommand{\source}[1]{
  \tiny
  \begin{textblock}{}[1, 0](1, 1)
    \hfill
    \usebeamercolor[fg]{normal text} #1
  \end{textblock}}

\newcommand{\sourceleft}[1]{
  \tiny
  \begin{textblock}{}[0, 1](0, 1)
    \hspace{1in + \oddsidemargin}
    \usebeamercolor[fg]{normal text} #1
  \end{textblock}}

%%%%%%%%%%%%
% TEMPLATE %
%%%%%%%%%%%%

% Remove beamer template
\setbeamertemplate{frametitle}
{\begin{centering}\smallskip
    \insertframetitle\par
    \smallskip\end{centering}}
% \setbeamertemplate{itemize item}{$\bullet$}
\setbeamertemplate{itemize item}{$\bullet$}
\setbeamertemplate{itemize subitem}{$\circ$}
\setbeamertemplate{itemize subsubitem}{$\circ$}
\setbeamertemplate{navigation symbols}{}
\setbeamercolor{normal text}{fg=black}

% itemize indents
\setlength{\leftmargini}{1.5em}
\setlength{\leftmarginii}{1.5em}
\setlength{\leftmarginiii}{1em}

% define a length for itemize indents
\newlength{\itemizeindent}
% locally override itemize indents to match descriptions like this
% \settowidth{\itemizeindent}{Geodesic:}
% \setbeamersize{description width=\itemizeindent}
% \setlength{\leftmargini}{\itemizeindent + \labelsep}

% endable to show layout
\usepackage{layout}

%%%%%%%%%%%%%%%%%%%%%%
% Centre all figures %
%%%%%%%%%%%%%%%%%%%%%%

\usepackage{letltxmacro}
% Save the meaning of \includegraphics
\usepackage{graphicx}
\LetLtxMacro\latexincludegraphics\includegraphics
% Update the include graphics command to include centering
\renewcommand{\includegraphics}[2][]%
    {\begin{center}
    \latexincludegraphics[#1]{#2}
    \end{center}}

%%%%%%%%%%%
% COLOURS %
%%%%%%%%%%%

% Define beamer colours
\definecolor{textcolour}{HTML}{000000}      % black
\definecolor{headingcolour}{HTML}{797979}   % grey
\definecolor{linkcolour}{RGB}{35,61,143}      % galaxy blue
\definecolor{bulletcolour}{RGB}{204,189,43}  % galaxy gold
\colorlet{paleheadingcolour}{headingcolour!50!}

% Place colours
\setbeamercolor{title}{fg=headingcolour}
\setbeamercolor{frametitle}{fg=headingcolour}
\setbeamercolor{normal text}{fg=textcolour}
\setbeamercolor{itemize item}{fg=bulletcolour}        % bullets
\setbeamercolor{itemize subitem}{fg=bulletcolour}     % sub-bullets
\setbeamercolor{itemize subsubitem}{fg=bulletcolour}  % sub-sub-bullets
\setbeamercolor{enumerate item}{fg=linkcolour}      % numbered lists
\setbeamercolor{enumerate subitem}{fg=linkcolour}
% \setbeamercolor{block title}{fg=headingcolour,bg=linkcolour}
% \setbeamercolor{block body}{fg=headingcolour,bg=linkcolour}
\setbeamercolor{alerted text}{fg=linkcolour}
\setbeamercolor{description item}{fg=linkcolour}

% Alert block
\setbeamertemplate{blocks}[rounded]
\setbeamercolor{block title}{fg=textcolour, bg=structure.bg}
\setbeamercolor{block body}{fg=textcolour, bg=structure.bg}
\setbeamercolor{block title alerted}{fg=linkcolour, bg=paleheadingcolour!75}
\setbeamercolor{block body alerted}{fg=textcolour, bg=paleheadingcolour!25}

% colour both section and url links
% see https://tex.stackexchange.com/a/401885
% \hypersetup{
%   colorlinks,
%   allcolors=.,
%   urlcolor=linkcolour,
% }

% note, url colouring is now clobbered by pandoc
% fix is to define the colours in the yaml as follows
% colorlinks: true
% linkcolor: linkcolour
% urlcolor: linkcolour

% or use this brute force approach
% Underline and colour all \hrefs
\LetLtxMacro\latexhref\href
\renewcommand{\href}[2]%
    {\textcolor{linkcolour}{\underbar{\latexhref{#1}{#2}}}}


%%%%%%%%%%%%%%%%%%%
% LATEX-ONLY COLS %
%%%%%%%%%%%%%%%%%%%

% not needed if you're using pandoc
% block syntax

% enable columns
% \def\begincols{\begin{columns}}
% \def\begincol{\begin{column}}
% \def\endcol{\end{column}}
% \def\endcols{\end{columns}}


%%%%%%%%%%%%%%%%%%%%%
% Set up title page %
%%%%%%%%%%%%%%%%%%%%%

% from https://tex.stackexchange.com/a/3927
% \usepackage{graphicx}   # already loaded above
\usepackage{tikz}

\setbeamertemplate{title page}{
\begin{tikzpicture}[remember picture,overlay]
  \node[at=(current page.north west),anchor=north west] {
    \latexincludegraphics[keepaspectratio,
                          height=0.3\paperheight]{img/Galaxy-Aust-Logo-Horizontal-RGB.pdf}
                          };
  % second logo if desired
  % \node[at=(current page.south west),anchor=south west] {
  %   \latexincludegraphics[keepaspectratio,
  %                         height=0.3\paperheight,width=0.3\paperwidth]{img/image2.png}
  %                         };
\end{tikzpicture}
\begin{center}
\vspace{0.1\paperheight}
\end{center}
\usebeamerfont{title}\inserttitle\par
\usebeamerfont{subtitle}\insertsubtitle\par
\usebeamerfont{date}\insertdate\par
\usebeamerfont{author}\insertauthor\par
\usebeamerfont{institute}\insertinstitute\par
}

%%%%%%%%%%%%%%
% TOC slides %
%%%%%%%%%%%%%%

% \setbeamercolor{section title}{fg=linkcolour,bg=white}
% \setbeamercolor{section}{fg=linkcolour,bg=white}
% \setbeamercolor{section name}{fg=linkcolour,bg=white}
\setbeamercolor{part title}{fg=linkcolour,bg=white}
\setbeamercolor{section in toc}{fg=linkcolour,bg=white}
\setbeamercolor{section in toc shaded}{fg=linkcolour,bg=white}
\setbeamertemplate{section in toc}{
  \inserttocsectionnumber.~\underbar{\inserttocsection}
}
\setbeamertemplate{section in toc shaded}{
  \inserttocsectionnumber.~\inserttocsection
}
\setbeamertemplate{section page}{
\begin{tikzpicture}[remember picture,overlay]
  \node[at=(current page.north west),anchor=north west] {
    \latexincludegraphics[keepaspectratio,
                          height=0.3\paperheight]{img/Galaxy-Aust-Logo-Horizontal-RGB.pdf}
};
\end{tikzpicture}
  \begin{center}
  \vspace{0.1\paperheight}
  \end{center}
  \usebeamerfont{title}\inserttitle\par
  \begin{beamercolorbox}[sep=12pt]{part title}
    \usebeamerfont{author}\tableofcontents[currentsection]
  \end{beamercolorbox}
}

\AtBeginSection[]{
\frame[plain]{\sectionpage}
}

%%%%%%%%%%%%%%%%%
% SLIDE NUMBERS %
%%%%%%%%%%%%%%%%%

% slide number footer
% \setbeamertemplate{footline}[frame number]

% title on slides
% \setbeamertemplate{headline}{
%   \ifnum \theframenumber=1
%   
%   \else
%   \vspace{1ex}
%   \hspace{1in + \hoffset + \oddsidemargin}
%   \usebeamercolor[fg]{normal text}
%   \insertdate :
%   \inserttitle
%   \hfill
%   \fi}

%%%%%%%%%%%%
% SI units %
%%%%%%%%%%%%

\usepackage[binary-units,unit-mode=text]{siunitx}
\AtBeginDocument{%
    \DeclareSIUnit\basepair{bp}%
}
\AtBeginDocument{%
    \DeclareSIUnit\base{base}%F
}

%%%%%%%%%%%%%%%
% MATHS FONTS %
%%%%%%%%%%%%%%%

% maths fonts with mathastext
% this works, but digits are rendered in the normal math font (not Lato)
% \usepackage[italic,defaultmathsizes,symbolgreek]{mathastext}
% \renewcommand{\familydefault}{\sfdefault}

% maths fonts with unicode-math
% allows you to change font, but *only* to a supported font
% https://www.ctan.org/pkg/unicode-math?lang=en
% \usepackage{unicode-math}
% \setmathfont{STIXTwoMath-Regular}

% Below is the "proper" way to do it. It works, in that digits and letters are
% rendered in Lato, but symbols get clobbered
% UPDATE: this works, but you need 
% --variable=mathspec in the pandoc arguments.
\usepackage{mathspec}
% \setmathfont(Digits,Latin,Greek)[Scale=MatchLowercase]{Lato}
% OR
\setallmainfonts(Digits,Latin,Greek)[Scale=MatchLowercase]{Lato}
% define a Var command for multi-character variables
% see https://tex.stackexchange.com/questions/482743/typesetting-multi-letter-variable-names-in-math-mode#comment1220353_482743
\newcommand\Var[1]{\mathrm{#1}}

%%%%%%%%%%%%%%%%%%%%%
% GRAPHS / NETWORKS %
%%%%%%%%%%%%%%%%%%%%%

% for graphs and networks
% see https://tex.stackexchange.com/a/85700
\makeatletter % undo the wrong changes made by mathspec
\let\RequirePackage\original@RequirePackage
\let\usepackage\RequirePackage
\makeatother
\usepackage{tikz-network}
\tikzset{
  network invisible/.code={%
    \def\VertexFillOpacity{0}
    \def\VertexLineOpacity{0}
    \def\VertexTextOpacity{0}
    \def\EdgeOpacity{0}
    \def\EdgeTextFillOpacity{0}
    \def\EdgeTextOpacity{0}},
  network visible on/.code={%
    \alt#1{}{\tikzset{network invisible}}}}

% colours for highlights
\definecolor{viridis1}{HTML}{440154}
\definecolor{viridis2}{HTML}{472D7B}
\definecolor{viridis3}{HTML}{3B528B}
\definecolor{viridis4}{HTML}{2C728E}
\definecolor{viridis5}{HTML}{21908C}
\definecolor{viridis6}{HTML}{27AD81}
\definecolor{viridis7}{HTML}{5DC863}
\definecolor{viridis8}{HTML}{AADC32}
\definecolor{viridis9}{HTML}{FDE725}

% hilight on slides in ovspec (#1), see beamer manual for \temporal
% use like this
% \begin{scope}[network hili2 on={<2->}{viridis3}]
\tikzset{
  network hili/.code={%
      \def\VertexFillColor{#1}
      \def\VertexLineColor{#1}
      \def\EdgeColor{#1}
      \def\TextColor{#1}%
  },
  network hili on/.code args={#1#2}{%
    \temporal#1{}{%
        \tikzset{network hili=#2}
    }{}}}

% tikzmark allows you to layout text notes on equations
% example: https://github.com/TomHarrop/comp90014_week05_ml/blob/80518394d6ce1da2bc6784caa6de4ea7c8c93560/week05a.Rmd#L514-L523
\usetikzlibrary{tikzmark}

%%%%%%%%%%%%%%%%%%%%%%
% INDENT CODE BLOCKS %
%%%%%%%%%%%%%%%%%%%%%%

% pandoc only defines \Shaded if it's used in the markdown
% otherwise you get an error
\ifdefined\Shaded
    \usepackage[framemethod=TikZ]{mdframed}
    
    % mdframed bug
    % https://tex.stackexchange.com/a/292090
    \usepackage{xpatch}
    \makeatletter
    \xpatchcmd{\endmdframed}
      {\aftergroup\endmdf@trivlist\color@endgroup}
      {\endmdf@trivlist\color@endgroup\@doendpe}
      {}{}
    \makeatother
    
    % https://stackoverflow.com/a/49937901
    
    \definecolor{shadecolor}{gray}{.95}
    \renewenvironment{Shaded}{\begin{mdframed}[
      backgroundcolor=shadecolor,
      linecolor = shadecolor,
      leftmargin=\leftmargini,
      innerleftmargin=0.5em,
      innerrightmargin=0.5em,
      innertopmargin=0.5ex,
      innerbottommargin=0.5ex,
      skipabove=2ex,
      skipbelow=2ex]}{\end{mdframed}}
\fi

%%%%%%%%%%%%%%
% Algorithms %
%%%%%%%%%%%%%%


\usepackage[lined,linesnumbered]{algorithm2e}

% argmax is pretty common in algorithms
\DeclareMathOperator*{\argmax}{arg\,max}
\DeclareMathOperator*{\argmin}{arg\,min}

%%%%%%%%%%%%%%%%%%%%%%
%%%%%%%%%%%%%%%%%%%%%%
